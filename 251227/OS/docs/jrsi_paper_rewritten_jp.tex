\documentclass[12pt,a4paper]{jsarticle}

% Packages
\usepackage[utf8]{inputenc}
\usepackage{amsmath}
\usepackage{amsfonts}
\usepackage{amssymb}
\usepackage[dvipdfmx]{graphicx}
\usepackage{float}
\usepackage{amsthm}
\usepackage[dvipdfmx]{hyperref}
\usepackage[numbers]{natbib}
\usepackage{geometry}
\geometry{margin=2.5cm}

% PDF出力設定
\hypersetup{
    colorlinks=true,
    linkcolor=blue,
    urlcolor=blue,
    citecolor=blue,
    bookmarks=true,
    bookmarksopen=false,
    pdftitle={Open-Sparsity: Geometric Organization of Void-Matter Morphology},
    pdfauthor={Kotaro Iwata}
}

% Theorem environments
\theoremstyle{definition}
\newtheorem{definition}{定義}[section]

% Title
\title{Open-Sparsity: 空間ネットワークにおける空隙--物質形態の幾何学的構成}
\author{岩田 幸太朗}
\date{\today}

\begin{document}

\maketitle

\begin{abstract}
従来の密度ベースの指標は、同一密度を持ちながら機能的に異なる空間パターンを区別できないことが多い。ラクナリティ、マルチフラクタル次元、パーコレーション閾値といった個別の指標は特定の形態的側面を捉えるが、異なる形態間の「構造的距離」を定量化する統一的な枠組みを欠いている。本論文では、\textbf{Open-Sparsity (OS)}を、新しい指標としてではなく、空隙と物質の配置を高次元特徴空間内の点として扱う\textbf{距離幾何学的枠組み}として導入する。多スケールの形態的特性から構築された距離空間を定義することで、密度が不変であっても構造的差異を厳密に定量化できることを示す。合成参照パターンを用いた検証により、このアプローチが解像度や確率的な変動に対して頑健であることを実証する。この枠組みは、都市計画、生態学、人工生命における複雑な空間システムを分析するための新しいレンズを提供する。
\end{abstract}

\section{序論}

\subsection{形態分析における「密度の罠」}
都市居住地、生体組織、生態学的景観など、複雑な空間システムの分析において、密度は主要な一次統計量である。しかし、密度は空間情報の「非可逆的な(lossy)」圧縮である。人口密度が全く同じ2つの都市でも、その構造は大きく異なる可能性がある。一方はコンパクトで単中心的なスプロール現象を示し、もう一方は断片化された多中心的なネットワークであるかもしれない。これらの構造的差異は、交通流、病気の伝播、生態系の回復力といった重要なシステム挙動を決定づける。

\subsection{既存指標の限界}
これらの非密度的な構造特性を捉えるために、研究者は様々な高度な指標を採用してきた:
\begin{itemize}
    \item \textbf{ラクナリティ (Lacunarity)}:空隙サイズの不均質性を定量化する。
    \item \textbf{パーコレーション理論 (Percolation theory)}:連結性と相転移を評価する。
    \item \textbf{マルチフラクタル解析 (MFA)}:質量分布のスケーリング挙動を捉える。
\end{itemize}

これらの指標はそれぞれ単独では有効であるが、通常は特定の特性を測定するために使用され、異なる形態間の\textit{関係性}を定義するためには使用されない。現在のアプローチには、\textbf{「パターンAはパターンBと構造的にどの程度異なるのか?」}という根本的な問いに答える統一的な「距離」メトリックが欠けている。

\subsection{Open-Sparsityフレームワーク}
本論文では\textbf{Open-Sparsity (OS)}を紹介する。OSが何でないかを述べることは極めて重要である:
\begin{quote}
\textbf{Open-Sparsityは新しい指標としてではなく、形態の幾何学的構成(geometric organization)として提案される。}
\end{quote}

形状を特徴づける単一の「魔法の数字」を探す代わりに、OSは直交する形態的特性(空隙の不均質性、連結性、質量スケーリング)に対応する軸を持つ高次元距離空間を定義する。この空間において、各空間パターンは「点」であり、点間のユークリッド距離はそれらの「構造的非類似性」を表す。

\subsection{貢献}
本研究の貢献は以下の通りである:
\begin{enumerate}
    \item \textbf{概念的貢献}:「疎(sparsity)」をスカラー値(低密度)としてではなく、空隙--物質配置の「距離幾何学」として再定義する。
    \item \textbf{方法論的貢献}:標準的な指標(ラクナリティ、MFA、パーコレーション)を用いてこの距離空間を構築するための頑健なプロトコルを提案する。
    \item \textbf{検証的貢献}:この幾何学的枠組みが、同一密度を持つ合成パターンを明確に区別し、確率的変動の下でも頑健であることを示しつつ、その失敗モード(適用限界)を明示的に特定する。
\end{enumerate}

\section{Open-Sparsityの概念的定義}

\subsection{「Sparse(疎)」から「Open(開)」へ}
伝統的に、「疎」は単に「密度が高くない」ことを意味する。しかし、複雑系においては、何もない空間(空隙)の\textit{配置}が物質そのものと同じくらい重要である。
\begin{itemize}
    \item \textbf{閉じた疎(Closed Sparsity)}:空隙が孤立し、分断されたポケット状になっている(例:スイスチーズ)。
    \item \textbf{開いた疎(Open Sparsity)}:空隙が相互に連結し、システム全体を横断できるチャネルを形成している。
\end{itemize}

我々は「Open-Sparsity」を、空隙相がトポロジカルに「開いており」(パーコレーションしており)、かつ構造的に不均質である状態として定義する。

\subsection{空隙--物質の幾何学}
これを測定するために、単一のスケールに頼ることはできない。以下の3点を捉える「距離」が必要である:
\begin{enumerate}
    \item \textbf{どの程度「穴だらけ」か?}(ラクナリティ)
    \item \textbf{どの程度「つながっている」か?}(パーコレーション閾値)
    \item \textbf{どの程度「複雑」なスケーリングか?}(マルチフラクタルスペクトラム幅)
\end{enumerate}

これらの3つの特性が我々の座標系の基底を形成する。パターンはその密度によってではなく、この\textit{OS空間}における座標によって定義される。

\section{合成パターンの設計}

我々の枠組みを厳密にテストするために、6つの異なる合成二値パターンを生成した。
\textbf{極めて重要な点として、すべてのパターンは同一の密度($d = 0.10$)を共有している。}
これにより、密度による差異を排除し、純粋な形態的・テクスチャ的差異を評価することができる。

\begin{enumerate}
    \item \textbf{Uniform(均一)}:グリッド状、規則的な配置。(最大の均質性の参照)
    \item \textbf{Random(ランダム)}:ポアソン分布。(確率的なベースラインの参照)
    \item \textbf{Clustered(クラスター化)}:物質の明確な塊。(高いラクナリティ)
    \item \textbf{Linear(線形)}:繊維状、道路のような構造。(異方的な連結性)
    \item \textbf{Radial(放射状)}:同心円状のリング。(中心集中型構造)
    \item \textbf{Multi-Nuclear(多核)}:複数の中心。(多中心構造)
\end{enumerate}

これらのパターンは、$512 \times 512$ピクセルの解像度で生成され、密度調整アルゴリズムにより厳密に$d=0.10$に制御されている。

\section{方法:距離幾何学の構築}

我々の方法論は、任意の2つのパターン$i$と$j$の間の距離行列$D_{ij}$を構築することに焦点を当てている。
本分析では、特に\textbf{Lacunarity(空隙性)}と\textbf{Multifractal Analysis (MFA)}を用いることで、単一のフラクタル次元では捉えきれない構造的差異を定量化する。

\subsection{Lacunarity Analysis ($\Lambda$)}
Lacunarityは、フラクタル集合の並進不変性からの逸脱を測定する指標であり、空隙(Lacunae)の分布の不均質性を定量化する\cite{allain1991characterization, plotnick1996lacunarity}。
計算にはGliding Box法を用い、ボックスサイズ$r$における質量の平均$\mu(r)$と分散$\sigma(r)$から次式で定義される:
$$ \Lambda(r) = \left(\frac{\sigma(r)}{\mu(r)}\right)^2 + 1 $$
$\Lambda$が高いほど、空隙が大きく偏在しており(クラスター性が強い)、低いほど均質であることを示す。

\subsection{Multifractal Analysis (MFA)}
都市のような複雑なシステムは、単一のスケーリング則(モノフラクタル)ではなく、異なる密度モーメントごとに異なるスケーリング則を持つことが多い\cite{hentschel1983infinite, frankhauser1998fractal}。
一般化次元$D_q$は次式で計算される:
$$ D_q = \frac{1}{q-1} \lim_{r \to 0} \frac{\log \sum_i p_i^q}{\log r} $$
ここで$q$はモーメントの次数であり、$q > 0$は高密度領域(中心部)を、$q < 0$は低密度領域(周辺部)を強調する。また、Legendre変換により特異性スペクトル$f(\alpha)$を導出し、その幅$\Delta \alpha$を構造的複雑性の指標として用いる。

\section{結果:Open-Sparsity空間の可視化}

\subsection{Chart A: テクスチャの識別(Lacunarity)}
図\ref{fig:charta}は、同一密度($d=0.10$)を持つ異なるパターン(Random, Linear, Clustered)のLacunarity曲線を示している。
密度が同一であるにもかかわらず、曲線は劇的に異なる挙動を示す。
\begin{itemize}
    \item \textbf{Random/Uniform}:平坦または急速に減衰し、均質であることを示す。
    \item \textbf{Clustered}:初期値が高く、減衰が遅い。これは大きな空隙が存在することを示唆する。
    \item \textbf{Linear}:特徴的なべき乗則に従う減衰を示す。
\end{itemize}
これは、Plotnickら\cite{plotnick1996lacunarity}が指摘した通り、Lacunarityが「スケール依存の異質性」を捉える強力なツールであることを確認するものである。

\begin{figure}[H]
\centering
\includegraphics[width=0.95\textwidth]{ChartA_Lacunarity.png}
\caption{同一密度におけるLacunarity曲線の比較。パターンの違い(テクスチャ)が曲線の形状として明確に表れている。}
\label{fig:charta}
\end{figure}

\subsection{Chart B: 構造的複雑性(MFA Spectrum)}
図\ref{fig:chartb}は、一般化次元スペクトル$D_q$と特異性スペクトル$f(\alpha)$を示す。
\begin{itemize}
    \item \textbf{Uniform}:スペクトルが狭く、モノフラクタルに近い単純な構造を持つ。
    \item \textbf{Multi-Nuclear / Clustered}:スペクトル幅$\Delta \alpha$が広く、強いマルチフラクタル性(階層的な構造)を持つことがわかる。
\end{itemize}
このスペクトル幅は、都市形態の「豊かさ」や階層性を反映していると考えられる。

\begin{figure}[H]
\centering
\includegraphics[width=0.95\textwidth]{ChartB_MFA.png}
\caption{Multifractal解析の結果。左:パターン例。右:$D_q$および$f(\alpha)$スペクトル。複雑なパターンほどスペクトル幅が広い。}
\label{fig:chartb}
\end{figure}

\subsection{Chart C: Open-Sparsity空間}
最後に、3つの指標(Lacunarity $r=16$, MFA Width $\Delta \alpha$, Percolation $r_{crit}$)によって張られる空間を図\ref{fig:chartc}に示す。
この空間において、各都市形態は連続的な分布を示し、離散的な分類(「線状」「多核」など)を超えた定量的な位置付けが可能となる。
特に、Lacunarity(空隙性)とMFA Width(複雑性)の軸によって、形態のタイプが明確に分離されていることがわかる。

\begin{figure}[H]
\centering
\includegraphics[width=0.95\textwidth]{ChartC_Space.png}
\caption{3つの指標(Lacunarity, MFA Width, Percolation)で構成されるOpen-Sparsity空間。異なる都市形態がこの距離空間内で明確に分離されている。}
\label{fig:chartc}
\end{figure}


\section{失敗モード}

万能な指標は存在しない。Open-Sparsityがいつ失敗するかを理解することは不可欠である。

\begin{itemize}
    \item \textbf{解像度による崩壊}:極端に低い解像度(例:$< 32^2$)では、特定の構造的特徴(「Linear」や「Radial」など)がエイリアシングを起こし、「Random」ノイズと区別がつかなくなる。図\ref{fig:failure}はこの崩壊を示している。
    \item \textbf{密度の支配}:密度が極端に異なる場合(例:$d=0.1$ vs $d=0.9$)、密度の効果が形態的効果を圧倒してしまう。OSは、比較可能な密度領域内のパターンを比較するために使用するのが最適である。
\end{itemize}

\begin{figure}[H]
\centering
%\includegraphics[width=0.9\textwidth]{../outputs/jrsi_figure1/figures/figure5_failure.png}
\caption{構造崩壊の失敗モード例。左:標準状態(構造が保存されている)。右:極低解像度($32 \times 32$)による構造崩壊。(※図は再生成中)}
\label{fig:failure}
\end{figure}

\section{考察}

「値」から「距離」へと焦点を移すことで、Open-Sparsityは密度が見逃してしまう微細な構造変化を定量化することを可能にする。

\textbf{実世界への示唆:都市のレジリエンス}
同じ人口密度を持つ2つの都市を考える。都市A(クラスター型)は高い局所効率を持つかもしれないが、接続の切断に対して脆弱である可能性がある。都市B(均一型)は頑健だが非効率かもしれない。OSは、特定の都市がこれらの原型の間のどこに位置するかを正確に測定する「定規」を提供し、計画者が単なる「密度目標」ではなく「失敗からの距離」を最適化することを可能にする。

人工生命(ALife)においては、この指標によって進化した構造の\textit{形態的多様性}を測定することができ、システムが単にピクセルを再配置しているだけでなく、真に新しい形態を生成していることを証明できる。

\section*{謝辞}
[ここに追加予定]

% References
\begin{thebibliography}{99}
\bibitem{allain1991characterization}
Allain, C., \& Cloitre, M. (1991). Characterization of the lacunarity of random and deterministic fractal sets. \textit{Physical Review A}, 44(6), 3552-3558.

\bibitem{hentschel1983infinite}
Hentschel, H. G. E., \& Procaccia, I. (1983). The infinite number of generalized dimensions of fractals and strange attractors. \textit{Physica D: Nonlinear Phenomena}, 8(3), 435-444.

\bibitem{plotnick1996lacunarity}
Plotnick, R. E., Gardner, R. H., Hargrove, W. W., Prestegaard, K., \& Peritt, M. (1996). Lacunarity analysis: A general technique for the analysis of spatial patterns. \textit{Physical Review E}, 53(5), 5461.

\bibitem{frankhauser1998fractal}
Frankhauser, P. (1998). Fractal geometry of urban patterns and their morphogenesis. \textit{Discrete Dynamics in Nature and Society}, 2, 127-145.

\end{thebibliography}

\end{document}
